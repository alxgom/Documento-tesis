	\begin{center}
		\section*{Fresnel Problem}
	\end{center}
	
\section{El problema transversal}

	
	Maxwell-bloch equation, after aproximations, with external phase modulation on $E_y$:
	
	Siguiendo los pasos del paper  \cite{Lugiato:90}
	
	Partiendo de 
	
	\[
	\begin{cases}
	\frac{1}{\nu} \partial_{\tau}F=i \frac{\delta \Omega'}{\nu}F -\partial_{\eta}F +\frac{i}{4}\nabla^2_{\bot} F -\alpha \Lambda P \\
	\partial_{\tau} P=-[FD + (1+i\Delta')P] \\
	\partial_{\tau} D=-\gamma_{||}(D-\chi(\rho)+\tfrac{1}{2}(F^*P+FP^*)) \\
	\end{cases}
	\]
	
	Obs: \[\tfrac{1}{2}(F^*P+FP^*)=\tfrac{1}{2}[(F_R-iF_I)(P_R+iP_I)+(F_R+iF_I)(P_R-iP_I)]=F_RP_R+F_IP_I\]
	
	
	$E(r,z,t)=\frac{\hbar \sqrt{ \gamma_{\bot} \gamma_{||} }}{2\mu } \frac{1}{2} [F(r,z,t) e^{i(k_0 z - w_0 t)} + c.c.]$\\
	
	$\delta \Omega'=\delta \Omega /\gamma_{\bot}$
	$\Delta'=\delta'_{ac}-\delta \Omega'$
	$\delta'_{ac}=\frac{w_a-w_0}{\gamma_{\bot}}$
	\textcolor{red}{en el paper dice que $m>0$}
	
	Se plantea una descomposición de Galerkin usando una base de funciones $A^i_{pm}$ tal que
	
	$F(\rho,\phi,\eta,\tau)=\sum_{p,m,i} A^i_{pm}(\rho,\phi,\tau) f^i_{pm}(\eta,\tau)$
	
	.....
	

	 
	 
	Después de hacer aprox..
	
	
	$\begin{cases}
	\partial_t E=-k\left([f(\rho)-i.\left[\delta+\frac{1}{2}a\left[\dfrac{\nabla^2_{\bot}}{4}+1-\rho^2\right]\right]].E-2CP\right)\\
	\partial_t P=-\gamma_{\bot}[(1+i\delta)P+E.D]\\
	\partial_t D=-\gamma_{\parallel}(D-\chi(\rho)-\frac{1}{2}(E^*P+EP^*))
	\end{cases}$
	
	
	with $E=E_x+E_y$ ;  $E_x , E_y \in \mathbb{C}$ and $P=P_x+P_y$ ; $P_x , P_y \in \mathbb{C}$
	
	Here $f(\rho)=5+4\tanh(5(\rho-\rho_0))$, where $\rho_0$ is the cavity effective width.
	$a=\dfrac{8}{T}\tan^{-1}(\frac{1}{4\eta_{1}})$ and $\eta_1=\dfrac{\pi \rho^2_0}{L\lambda}$.
	
	$\chi(\rho)=e^{(-1.2 \frac{\rho}{\rho_0})^2}\dfrac{[1+e^{-\rho_0^2}]}{[1+e^{(\rho^2-\rho_0^2)}]}$

	
	$F(\rho,\phi,\eta,\tau)=\sum_{p,m,i} A^i_{pm}(\rho,\phi) \psi^i_{pm}(\tau)$\\
	$P(\rho,\phi,\eta,\tau)=\sum_{p,m,i} A^i_{pm}(\rho,\phi) p^i_{pm}(\tau)$\\
	$D(\rho,\phi,\eta,\tau)=\sum_{p,m,i} A^i_{pm}(\rho,\phi) d^i_{pm}(\tau)$\\
	
	con 
	
	\[ A^i_{pm}(\rho,\phi)=2(2\rho^2)^{m/2}[\frac{p!}{(p+m)!}]^{1/2}L^m_p(2\rho^2)e^{-\rho^2}B^i_{m}(\phi)   \]
	donde 

	\[B^i_m(\phi)=
	\begin{cases}
	\frac{1}{\sqrt{2\pi}} \qquad \textit{si $m=0$} \\
	\frac{1}{\sqrt{\pi}}\sin(m\phi) \qquad \textit{si $m>0$, i=1}\\
	\frac{1}{\sqrt{\pi}}\cos(m\phi) \qquad \textit{si $m>0$, i=2} \\
	\end{cases}
	\]

	
	Con esta definición, los $A^i_{pm}$ con un conjunto ortonormal bajo el producto interno, como se muestra en \ref{PI}
	\[\int_{0}^{\infty}\int_{0}^{2\pi} A^{i*}_{pm} A^{i'}_{p'm'} \rho d\rho d\phi=\delta_{i i'}\delta_{m m'}\delta_{p p'}    \]

	\textcolor{red}{Poner la dem del PI}
	\todo{Poner la dem del PI}
	
	
		
	\subsection{Descomposición de Galerkin}
	
		
		Galerkin BON, the Gauss-Laguerre polynomials:
		
		$$ A_{pm}(\rho, \varphi)=2(2\rho^2)^{m/2}(\dfrac{p!}{(p+m)!})^{1/2}e^{-\rho{^2}}L_p^m(2\rho{^2})e^{im\varphi} $$
		
		So, let $f$ be a initial condition for some variable,  $f$ can now be written in the Gauss-Laguerre space as a linear combination, by proyecting it (such as a Fourier Series):
		
		$$f=\sum_{pm}C_{pm}A_{pm}$$
		
		where $$C_{pm}(t)=\langle f | A_{pm}\rangle=\iint f.A_{pm}\rho \, d\!\rho d\!\varphi  $$.
		
		Following the paper  http://sci-hub.io/10.1364/JOSAB.7.001019 ..
		
		we use the internal product  $$\langle A_{p'm'} | A_{pm}\rangle=\iint A_{p'm'}^*.A_{pm}\rho \, d\!\rho d\!\varphi=\delta_{\rho \rho'}\delta_{\varphi \varphi'}\delta_{i i'}  $$.
		
		
		so, if we consider 
		
		$\begin{cases}
		E(\rho, \varphi, t)=\sum_{pm}\psi_{pm}(t)A_{pm}, \qquad \text{with } \qquad \psi_{pm}=\iint E(t).A_{pm}\rho \, d\!\rho d\!\varphi \\
		P(\rho, \varphi, t)=\sum_{pm}p_{pm}(t)A_{pm}, \qquad \text{with } \qquad p_{pm}=\iint P(t).A_{pm}\rho \, d\!\rho d\!\varphi \\
		D(\rho, \varphi, t)=\sum_{pm}d_{pm}(t)A_{pm}, \qquad \text{with } \qquad d_{pm}=\iint D(t).A_{pm}\rho \, d\!\rho d\!\varphi \\
		\end{cases}$
		
		Replacing in the differential equations, and taking the internal product with $A_{p'm'}$, we arrive at (after some redefinitions)
		
		$\begin{cases}
		\partial_t \psi_{pm}=-k'\left([f(\rho)-\left[i(\delta +a)+1 \right]\psi_{pm}-2Ck'p_{pm}\right)\\
		\partial_t p_{pm}=-\gamma_{\bot}[(1+i\delta)p_{pm}+\sum\sum \Gamma (\sigma, \sigma', \sigma'') \psi_{p'm'}d_{p''m''}]\\
		\partial_t d_{pm}=-\gamma_{\parallel}(d_{pm}-\chi(\rho)_{pm}-\frac{1}{2}\sum\sum [\Gamma (\sigma, \sigma', \sigma'') \psi_{p'm'}^*p_{p''m''} + c.c.])
		\end{cases}$
		
		
		$$\Gamma (\sigma, \sigma', \sigma'')=\iint A_{pm}A_{p'm'}A_{p''m''}\rho \, d\!\rho d\!\varphi  $$(según el paper)
		
	Obs:
		Los términos que están solo, quedan solos ya que:
		
		\begin{equation}
			F=\sum_{p'm'}\psi_{p'm'}A_{p'm'}%=(\iint f.A_{pm}\rho \, d\!\rho d\!\varphi) A_{pm}.
		\end{equation}
		
		por lo tanto 
		
		\begin{equation}
			\iint F.A^*_{pm}\rho \, d\!\rho d\!\varphi=\sum_{p'm'} \psi_{p'm'} \iint A_{p'm'}.A^*_{pm}\rho \, d\!\rho d\!\varphi =\sum_{p'm'} \psi_{p'm'} \delta_{pp'}\delta_{mm'}=\sum_{pm} \psi_{pm}
		\end{equation}
		
		Mientras que los términos no lineales como $FD$:
		
		\begin{equation}
			FD=\sum_{p'm'}\sum_{p''m''}\psi_{p'm'}d_{p''m''}A_{p'm'}A_{p''m''}%=(\iint f.A_{pm}\rho \, d\!\rho d\!\varphi) A_{pm}.
		\end{equation}
	
		Al aplicar el producto interno:
		
		\begin{equation}
			\iint FD.A^*_{pm}\rho \, d\!\rho d\!\varphi=\sum_{p'm'}\sum_{p''m''}\psi_{p'm'}d_{p''m''} \iint A_{p''m''} A_{p'm'}.A^*_{pm}\rho \, d\!\rho d\!\varphi=\sum_{p'm'}\sum_{p''m''}\psi_{p'm'}d_{p''m''}\Gamma (\sigma, \sigma', \sigma'')
		\end{equation}
		
		Por lo tanto en la integral de $\Gamma$, queda una parte compleja. Lo cual rompe una de las simetrias de permutacion en $i$.

		\begin{equation}
		\Gamma (\sigma, \sigma', \sigma'')= \iint A_{p''m''} A_{p'm'}.A^*_{pm}\rho \, d\!\rho d\!\varphi
		\end{equation}

	y la parte angula cumple con la relacion de ortogonalidad :
	
	\begin{equation}
		\int e^{i(m'+m''-m)}d\!\varphi=cte\, \delta(m'+m''-m)
	\end{equation}

	\[B^i_m^*(\phi)=
	\begin{cases}
	\frac{1}{\sqrt{2\pi}} \qquad \textit{si $m=0$} \\
	-\frac{1}{\sqrt{\pi}}\sin(m\phi) \qquad \textit{si $m>0$, i=1}\\
	\frac{1}{\sqrt{\pi}}\cos(m\phi) \qquad \textit{si $m>0$, i=2} \\
	\end{cases}
	\]

	\section{Code deduction:}
	
	From the paper 'Symmetry breaking, dynamical pulsations, and turbulence in the transverse intensity patterns of a laser: the role played by defects', we use the Galerkin method to integrate the equations.
	
	The galerkin method consist in choosing a function basis that is compatible with the problem (here the bondary conditions are implicit in the choice made for the basis), expanding some initial conditions in the basis, and then evolving the basis functions.
	
	Here we are under the hypotesis that we can perform variable separation under the cylindrical simmetry of the boundary conditions ($\rho$ decays to 0 at infinity).
	
	
	Thus the problem translates from a $n$ PDE equations to $m.n$ ordinary equations, where $m$ is the degree in witch i expand the function.
	
	
	For this problem the chosen funcions are the Gauss-Laguerre polynomials:
	
	\[A_{pm}(\rho, \varphi)=2(2\rho^2)^{m/2}(\dfrac{p!}{(p+m)!})^{1/2}e^{-\rho{^2}}L_p^m(2\rho{^2})e^{im\varphi} \] 
	
	where $-p<m<p$ 
	
	OBS: from \href{https://www.wikiwand.com/en/Gaussian_beam}{wikipedia: Gaussian Beam}
	
	"Beam profiles which are circularly symmetric (or lasers with cavities that are cylindrically symmetric) are often best solved using the Laguerre-Gaussian modal decomposition.[3] These functions are written in cylindrical coordinates using Laguerre polynomials. Each transverse mode is again labelled using two integers, in this case the radial index $p\ge 0$ and the azimuthal index l which can be positive or negative (or zero).
	
	\[ u(r,\phi,z)=\frac{C^{LG}_{lp}}{w(z)}\left(\frac{r \sqrt{2}}{w(z)}\right)^{\! |l|} \exp\! \left(\! -\frac{r^2}{w^2(z)}\right)L_p^{|l|}  \! \left(\frac{2r^2}{w^2(z)}\right)  \times \exp \! \left(\! - i k \frac{r^2}{2 R(z)}\right) \exp(i l \phi) \, \exp (   \! -ikz)  \,  \exp(i \psi(z)) \; \;   \]
	
	where $L_p^l$ are the generalized Laguerre polynomials.
	$ C^{LG}_{lp} $is a required normalization constant not detailed here; $w(z)$ and $R(z)$ have the same definitions as above. As with the higher-order Hermite-Gaussian modes the magnitude of the Laguerre-Gaussian modes' Gouy phase shift is exagerated by the factor $N+1$:
	
	$\psi(z)  = (N+1) \, \arctan \left( \frac{z}{z_\mathrm{R}} \right) $
	
	where in this case the combined mode number $N = |l| + 2p$. As before, the transverse amplitude variations are contained in the last two factors on the upper line of the equation, which again includes the basic Gaussian drop off in r but now multiplied by a Laguerre polynomial. The effect of the rotational mode number l, in addition to affecting the Laguerre polynomial, is mainly contained in the phase factor $exp(-il\phi)$, in which the beam profile is advanced ( or retarded) by l complete $2\pi $ phases in one rotation around the beam (in $\phi$). This is an example of an optical vortex of topological charge l, and can be associated with the orbital angular momentum of light in that mode."
	
	
	La idea aca es que quiero descomponer mi solucion en una base de soluciones(polinomiales, como en una cuadratura) de funciones que en la parte radial decaigan con $\rho \longrightarrow \infty$.
	

	
	Using rotation simetry, we use a trigonometric base for the angular space. Therefore we define : 
	\[\hat{A}_{pm}(\rho,\varphi)=K*R_{pm}(\rho)e^{im\varphi}\]
	
	
	$\left \langle A_p^m | A_{p'}^{m'} \right \rangle = \int_0^\infty A_p^m(x,\varphi) A_{p'}^{m'}(x,\varphi) dx d\varphi = \delta_{pp'}\delta_{mm'}$
	
	So, let $f$ be a initial condition for some variable,  $f$ can now be written in the Gauss-Laguerre space as a linear combination, by proyecting it (such as a Fourier Series):
	
	\[f=\sum_{pm}C_{pm}A_{pm}\]
	
	where \[C_{pm}=\langle f | A_{pm}\rangle=\iint f(x,\varphi).A_{pm}(x,\varphi) \, d\!x d\!\varphi  \].
	
	Since in our problem, we use $A_{pm}(2\rho^2)$ and ..
	
	So now, all our equations will be ordinary differential equations, such that 
	\[\partial_t f= \partial_t \sum_{pm}C_{pm}A_{pm}= \hat{\alpha} \sum_{pm}C_{pm}A_{pm} +\hat{L}\sum_{pm}C_{pm}A_{pm} + \hat{NL}\sum_{pm}C_{pm}A_{pm}    \]
	
	where $\hat{L}$ and $\hat{NL} $ are Linear and Non Linear operators.
	
	Therefore 
	
	\[\partial_t f= \sum_{pm}C_{pm} \partial_t A_{pm}= \sum_{pm}C_{pm} \hat{\alpha} A_{pm} +\sum_{pm} \hat{L}C_{pm}A_{pm} + \hat{NL}\sum_{pm}C_{pm}A_{pm} \].
	
	In our problem, $\hat{L}$ will be related with the laplacian operator in cylindrycal coordinates.
	
	So now, we have 3 problems to solve for developing our code: 
	
	1- Integrating numerically $\iint f.A_{pm}\rho \, d\!\rho d\!\varphi $ for each initial condition $f$ that we choose, and choosing the amount of G-L polynomials we will use in this approximation to make it as accurate as we need.
	
	2- Find the ordinary differential eqs we are going to use, and evolve them we some time method of our liking (Runge-kutta 4).
	
	3- Choosing some time and spatial steps for the integration.
	
%	\section{Coefficient integration}
	
%	\section{ODE deduction}
%	
%	First we wil recall the laplacian in spherical coordinates:
%	
%	\[ \nabla_{\bot}=\frac{1}{\rho}\partial_{\rho}(\rho\partial_{\rho}) +\frac{1}{\rho}\partial^2_{\varphi} =\partial^2_{\rho}+\frac{1}{\rho}\partial_{\rho} +\frac{1}{\rho^2}\partial^2_{\varphi}  \]
%	
%	So we start we the most troublesome part, $\nabla_{\bot}\sum_{pm}C_{pm}A_{pm}=\sum_{pm}C_{pm}\nabla_{\bot}A_{pm}$
%	
%	\[\nabla_{\bot}A_{pm}=\nabla_{\bot}R_{pm}e^{im\varphi}=\frac{1}{\rho}\partial_{\rho}(\rho\partial_{\rho}R_{pm}e^{im\varphi}) +\frac{1}{\rho}\partial^2_{\varphi}R_{pm}e^{im\varphi}=\frac{1}{\rho}\partial_{\rho}(\rho\partial_{\rho}R_{pm})e^{im\varphi} +\frac{-m^2}{\rho^2}R_{pm}e^{im\varphi}\]
%	
%	taking a look at $\frac{1}{\rho}\partial_{\rho}(\rho\partial_{\rho}R_{pm})$:
%	
%	%\[\frac{1}{\rho}\partial_{\rho}(\rho\partial_{\rho}R_{pm})=\dots= 2(2\rho^2)^{m/2}(\dfrac{p!}{(P+m!)})^{1/2}e^{\rho^2}[L^m_p(2\rho^2)(4\rho^2-2\rho-8+2/\rho)+\partial_\rho L^m_p()(16-16\rho^2)+4\rho\partial^2_\rho L^m_p()] \]
%	
%	ahora usamos que  \[\partial^k_x L^m_p(x)=(-1)^kL_{p-k}^{m+k}(x)\]
%	
%	entonces \[\begin{cases}
%	\partial_{\rho}L_p^m(2\rho^2)=-L_{p-1}^{m+1}(2\rho^2)4\rho\\
%	\partial^2_{\rho}L_p^m(2\rho^2)=L_{p-2}^{m+2}(2\rho^2)16\rho^2-4.L_{p-1}^{m+1}(2\rho^2)
%	\end{cases}
%	\]
%	
%	\[
%	\frac{1}{\rho}\partial_{\rho}(\rho\partial_{\rho}R_{pm})=\\
%	
%	2(2\rho^2)^{m/2}(\dfrac{p!}{(p+m)!})^{1/2}e^{-\rho{^2}}
%	\times \left[L_p^m(2\rho^2)(\frac{2}{\rho^2}+\frac{2}{\rho}-12+4\rho^2)+ \partial_{\rho}L_p^m(2\rho^2)(\frac{5}{\rho}-4\rho)+ \partial^2_{\rho}L_p^m(2\rho^2)      \right]
%	\]
%	
%	lo que queda como 
%	
%	\[
%	\frac{1}{\rho}\partial_{\rho}(\rho\partial_{\rho}R_{pm})=
%	
%	2(2\rho^2)^{m/2}(\dfrac{p!}{(p+m)!})^{1/2}e^{-\rho{^2}} 
%	\times  \left[L_p^m(2\rho^2)(\frac{2}{\rho^2}+\frac{2}{\rho}-12+4\rho^2) - L_{p-1}^{m+1}(2\rho^2)(24-16\rho^2)+  L_{p-2}^{m+2}(2\rho^2)16\rho^2  \right]
%	\]
%	
%	y por ultimo, usando la propiedad \href{https://www.wikiwand.com/en/Laguerre_polynomials}{wiki laguerre polynomials}
%	
%	$
%	L_m^{p}(x)= \frac{m+1-x}{p}  L_{p-1}^{(m+1)}(x)- \frac{x}{n} L_{p-2}^{(m+2)}(x)
%	$
%	
%	\[
%	\frac{1}{\rho}\partial_{\rho}(\rho\partial_{\rho}R_{pm})=
%	
%	2(2\rho^2)^{m/2}(\dfrac{p!}{(p+m)!})^{1/2}e^{-\rho{^2}} 
%	
%	\times  \left[ L_{p-2}^{m+2}\!(2\rho^2) \,(16\rho^2-\frac{2\rho^2}{p})   - L_{p-1}^{m+1}\!(2\rho^2) \, [(24-16\rho^2)+\frac{m+1-2\rho^2}{p}(\frac{2}{\rho^2}+\frac{2}{\rho}-12+4\rho^2)]  \right]
%	\]
%	
%	
%	Por lo tanto nos queda una ecuación del tipo:
%	
%	\[ \nabla_{\bot}A_{pm}=\nabla_{\bot}B_{pm}L_p^m e^{im\varphi}=B_{pm}[\alpha L_{p-2}^{m+2} - \beta L_{p-1}^{m+1} ] e^{im\varphi} -\frac{m^2}{\rho^2}B_{pm}L^m_p e^{im\varphi}
%	\]
%	
%	y repitiendo el truco anterior obtenemos 
%	\[ \nabla_{\bot}A_{pm}=\nabla_{\bot}B_{pm}L_p^m e^{im\varphi}=B_{pm}[\hat{\alpha} L_{p-2}^{m+2} - \hat{\beta}L_{p-1}^{m+1} ] e^{im\varphi}
%	\]
%	
%	con
%	
%	\[\begin{cases}
%	B_{pm}=2(2\rho^2)^{m/2}(\dfrac{p!}{(p+m)!})^{1/2}e^{-\rho{^2}} \\
%	\hat{\alpha}=(16\rho^2-\frac{2\rho^2-\frac{m^2}{\rho^2}}{p})\\
%	\hat{\beta}= (24-16\rho^2)+\frac{m+1-2\rho^2}{p}(\frac{2}{\rho^2}+\frac{2}{\rho}-12+4\rho^2+\frac{m^2}{\rho^2})
%	\end{cases}
%	
%	\]
%	
	
%	
%	Obs:
%	Si se quieren realizar los productos internos entre $\langle A_{p'm'i'} | A_{pmi}\rangle$ con $p=[0,N]$ y $p'=[0,M]$, la cantidad de integraciones que hay que realizar es:
%	
%	Ya que $p=[0,N]$ , la cantidad de funciones es $\sum_{k=0}^{N}(k+1)^2-(N+1)$. Donde el termino $(k+1)$ se debe a que los numeros de los polinomios empiezan desde 0 y no desde 1, el $2$ y el  $-(N+1)$ se debe a que por cada $m$ hay 2 valores de $i$ menos para $m=0$ en el que  $i$ toma un solo valor.
%	
%	Por lo tanto 
%	\[   \sum_{p=0}^{N}\sum_{p'=0}^{M} \# \langle A_{p'm'i'} | A_{pmi}\rangle= (\sum_{k=0}^{N}(k+1)^2-(N+1))(\sum_{j=0}^{M}(j+1)^2-(M+1))=
%	(\sum_{k=1}^{N+1}k^2-(N+1))(\sum_{j=1}^{M+1}j^2-(M+1))
%	\]
%	
%	Usando que si llamo $R=N+1$ vale que $\sum_{k=1}^{R}k^2=\frac{R(R+1)(2R+1)}{6}$ se puede reescribir como :
%	
%	\[\sum_{k=0}^{N}(k+1)^2-(N+1)=\sum_{k=1}^{R}2R^2-R=2R[\frac{2R^2+3R+1}{6}-1]\]	
%	
%	y con $R'=M+1$:
%	
%	\[ \sum_{p=0}^{R-1}\sum_{p'=0}^{R'-1} \# \langle A_{p'm'i'} | A_{pmi}\rangle=4RR'[\frac{2R^2+3R+1}{6}-1][\frac{2R'^2+3R'+1}{6}-1] \]
%	
%	
%	
%	Con el mismo razonamiento para $\Gamma(\sigma, \sigma', \sigma'' ) $, la cantidad de integrales a realizar es 
%	\[(\sum_{k=1}^{N+1}k^2-(N+1))(\sum_{j=1}^{N+1}j^2-(N+1))(\sum_{l=1}^{N+1}l^2-(N+1))=8R^3[\frac{2R^2+3R+1}{6}-1]^3\]
%	
%	Si $N \gg 1$. $\#\Sigma \approx 6,4 R^9$
	