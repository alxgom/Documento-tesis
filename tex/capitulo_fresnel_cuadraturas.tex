\section{Cuadraturas}
	\subsection{Cuadraturas de Gauss-Laguerre}
	
	Cuadratura de Gauss-Laguerre:
	
	Utiliza las $n$ raíces ($x_i$) de los Polinomios $L_n(x)$ , para aproximar integrales del tipo
	\[ \int_{0}^{\infty}f(x)e^{-x}dx\approx \sum_{i=1}^n w_n(x_i) f(x_i)   \]
	
	con 
	\[w_n(x_i)=\frac{x_i}{(n+1)^2[L_{n+1}(x_i)]^2}\]
	
	Cuadratura de Gauss-Laguerre asociada:
	
	Utiliza las $n$ raíces ($x_i$) de los Polinomios $L_n^{\alpha}(x)$ , $\alpha > -1$, para aproximar integrales del tipo
	\[ \int_{0}^{\infty}f(x)x^{\alpha}e^{-x}dx\approx \sum_{i=1}^n w^{\alpha}_n(x_i) f(x_i)    \]
	
	con 
	\[w^{\alpha}_n(x_i)=\frac{x_i \Gamma(n+\alpha +1)}{n!(n+1)^2[L^{\alpha}_{n+1}(x_i)]^2}\]
	
	
	\subsection{Implementaciones}
		
		Se realiza una comparacion de los métodos de convergencia para dos implementaciones de la cuadratura de Gauss-Laguerre.
		\begin{lstlisting}
		import scipy.special as sc
		
		def Bon_rl_genlaguerre(p,pp,m,mm,deg,alpha):
			xl,yl=sc.la_roots(deg,alpha)
			bon=sum(Rl(p,m,xl)*Rl(pp,mm,xl)*yl*np.exp(xl))
			return bon
			
		def Bon_rl_laguerre(p,pp,m,mm,deg):
			xl,yl=sc.l_roots(deg)
			bon=sum(Rl(p,m,xl)*Rl(pp,mm,xl)*yl*xl*np.exp(xl))
			return bon    
			
		\end{lstlisting}
		
		
		\begin{figure}[h]
			\includegraphics[width= 0.5\linewidth]{../../2d maxw galerkin/2d_maxwellblock_gallerkin/Python/Imagenes/comp_quad1.png}
			\includegraphics[width= 0.5\linewidth]{../../2d maxw galerkin/2d_maxwellblock_gallerkin/Python/Imagenes/comp_quad2.png}
			\caption{Comparación de convergencia de  Cuadraturas. Se realiza una integracion del Pi entre $R_5^3$ y $R_4^3$, utilizando una cuadratura de Gauss-Laguerre, una cuadratura de Gauss-Laguerre asociado con $\alpha =1$, y el metodo $scipy.quad$ (no iterativo). El Resultado teórico es $0$. }
		\end{figure}
		
		Pero si se tiene en cuenta que la hacer el producto interno la integral a resolver se puede reescribir haciendo un cambio de variables $x=2\rho^2$:
		
		\begin{align*}
			 \int_{0}^{\infty}R^{m}_{n}(\rho)R^{m'}_{n'}(\rho)\rho d\rho &=
			 \int_{0}^{\infty} 4\frac{n!}{(n+m)!}\frac{n'!}{(n'+m')!}L^{m}_{n}(x)L^{m'}_{n'}(x)\tfrac{1}{4}x^{(m+m')\tfrac{1}{2}}e^{-x}dx \\
			 & \approx \sum_{i=1}^N w_N^{(m+m')\tfrac{1}{2}}  \frac{n!}{(n+m)!}\frac{n'!}{(n'+m')!}L^{m}_{n}(x_i)L^{m'}_{n'}(x_i) 
		\end{align*}
			
		ahora puedo usar la cuadratura con los polinomios asociados de Laguerre con $\alpha=\tfrac{1}{2}(m+m')$.
		De esta manera los $x_i$ en los que evaluó a los $R$ también son sus raíces.  
		
		\begin{lstlisting}
			def Rl_bon(p,m,x):
			if m==0:
				m=float(m)
				Rl=2*sc.assoc_laguerre(x,p,k=m)
			elif m>0:
				m=float(m)
				Rl=2*np.sqrt(np.math.factorial(p)*(1./np.math.factorial(p+m)))*sc.assoc_laguerre(x,p,k=m) 
			return Rl
			
			def Bon_rl_genlaguerremod(p,pp,m,mm,deg):
				alpha=(m+mm)*.5    
				xl,yl=sc.la_roots(deg,alpha)    
				bon=sum(Rl_bon(p,m,xl)*Rl_bon(pp,mm,xl)*yl*.25)
				return bon
		\end{lstlisting}

		\begin{figure}[h]
			\includegraphics[width= 0.5\linewidth]{../../2d maxw galerkin/2d_maxwellblock_gallerkin/Python/Imagenes/comp_quad3.png}
			\caption{Comparación de convergencia de  Cuadraturas. Se realiza una integracion del Pi entre $R_5^3$ y $R_4^3$, utilizando una cuadratura de Gauss-Laguerre, una cuadratura de Gauss-Laguerre asociado con $\alpha =\tfrac{1}{2}(m+m')$. El Resultado teórico es $0$. }
		\end{figure}
		
		Se puede ver que con este método la convergencia es mucho mas rápida, de hecho ya es satisfactoria para $n=6$.
		
			\begin{center}
				\begin{table}[h]
					\begin{tabular}{|c|c|c|}
						\hline
						orden  		& 	       cuadratura  \Ts \Bs         \\ \hline
						1	&        		$-0.048187088154946753$  					  \\ \hline
						2	&        		$0.15058465048420852$					  \\ \hline
						3	&        		$0.40657855630736328$					  \\ \hline
						4	&        		$-6.9891926590402884e-17$					  \\ \hline
						5	&        		$1.9090683924971627e-16$					  \\ \hline
						6	&        		$3.9412917374193057e-15$					  \\ \hline	
						7	&        		$3.3306690738754696e-16$		             \\ \hline 					  								  				  								  
					\end{tabular}
					\caption{Valores para $\langle R_5^3 R_4^3 \rangle$ usando la cuadratura modificada de Gauss-Laguerre con los polinomios asociados con $\alpha =\tfrac{1}{2}(m+m')$ para distintos ordenes. Usando Scypi.quad se obtiene $-1.4220309711738635e-16$. }
					\label{tab: cuad gen-lag-mod}
				\end{table}
			\end{center}
		
		Se verifico realizando el producto interno entre $R$ y $R'$ para todos las funciones entre $n=[0,4]$ y $m=[0,4]$ utilizando este método con un orden de $p+p'+1$ y se obtuvieron los mismos resultados (con la misma cantidad de decimales) que usando la funcion scipy.quad .
		
		\begin{figure}[h]
			\includegraphics[width= 0.5\linewidth]{../../2d maxw galerkin/2d_maxwellblock_gallerkin/Python/Imagenes/comp_quad4.png}
			\caption{Comparación de convergencia de  Cuadraturas. Se realiza una integracion del Pi entre $R_{40}^3$ y $R_{40}^3$, utilizando una cuadratura de Gauss-Laguerre, una cuadratura de Gauss-Laguerre asociado con $\alpha =\tfrac{1}{2}(m+m')$. El Resultado teórico es $1$. }
		\end{figure}
		
		Se puede ver que con este método la convergencia es mucho mas rápida, de hecho ya es satisfactoria para $n=41$.
		
		De la misma forma se puede mdificar la cuadratura usual de Gauss-Laguerre
		
		
		\begin{figure}[h]
			\includegraphics[width= 0.5\linewidth]{../../2d maxw galerkin/2d_maxwellblock_gallerkin/Python/Imagenes/comp_quad5.png}
			\caption{Comparación de convergencia de  Cuadraturas. Se realiza una integracion del Pi entre $R_{40}^3$ y $R_{40}^3$, utilizando una cuadratura de Gauss-Laguerre, una cuadratura de Gauss-Laguerre asociado con $\alpha =\tfrac{1}{2}(m+m')$. El Resultado teórico es $1$. }
		\end{figure}
	
		\begin{figure}[h]
			\includegraphics[width= 0.5\linewidth]{../../2d maxw galerkin/2d_maxwellblock_gallerkin/Python/Imagenes/comp_quad6.png}
			\caption{Comparación de convergencia de  Cuadraturas. Se realiza una integracion del Pi entre $R_{30}^{25}$ y $R_{30}^{25}$, utilizando una cuadratura de Gauss-Laguerre, una cuadratura de Gauss-Laguerre asociado con $\alpha =\tfrac{1}{2}(m+m')$. El Resultado teórico es $1$.
			cv. asociados=31, cv. comun=43 }
		\end{figure}
		
		Para integrales con valores de $m$ mas elevados, la cuadratura con los polinomios asociados muestra una convergencia mas rapida. 
		
	\subsection{Cuadratura de Gauss-Legendre}
		
		Utiliza las $n$ raíces ($x_i$) de los Polinomios de Legendre $P_n(x)$ , para aproximar integrales del tipo
		\[ \int_{-1}^{1}f(x)dx\approx \sum_{i=1}^n w_n(x_i) f(x_i)    \]
		
		con 
		\[w_n(x_i)=\frac{2}{(1-x_i^2)[P_n'(x_i)]^2}\]
		
		Para trasladar el intervalo de integración a $[0,2\pi]$
		\[ \int_{0}^{2\pi}f(x)dx\approx \sum_{i=1}^n w_n(x_i) f(\hat{x}_i) \pi    \]
		\[ \hat{x}_i=(x_i+1)\pi \] 
		
	\subsection{Cuadraturas en 2d}
		
		Para realizar las integrales dobles, vamos a utilizar una cuadratura de Gauss-Laguerre en la dirección $\rho$ y una cuadratura de Gauss-Legendre en $\varphi$, de manera tal que 
		
		\[\int_{0}^{2\pi}\int_{0}^{\infty}f(\rho,\varphi)e^{-\rho}\rho d\rho d\varphi \approx \sum_{i=1}^n \sum_{i'=1}^{n'} w_n \hat{w}_{n'} f(\rho_i,\varphi_{i'})\rho_i \pi 
		\]
		
		con $w_n$ los pesos de la cuadratura de G-Laguerre a orden $n$  y $\hat{w}_{n'}$ los pesos de la cuadratura de G-Legendre a orden $n'$.
		
	\subsection{Integral de $\Gamma$}
		
		Para chequear los resultados obtenidos puedo comparar con una funcion facil de integrar:
		
		\begin{align*}
			\Gamma_{001001001} &= \int_{0}^{2\pi}\int_{0}^{\infty}A_{001}A_{001}A_{001}\rho d\rho d\varphi \\
			&=2\pi (2\pi)^{-3/2} \int_{0}^{\infty}8(e^{-2\rho^2})^{3/2}\rho d\rho\\
			&=2(2\pi)^{-1/2} \int_{0}^{\infty}(e^{-1,5 x}) dx\\
			&=(2\pi)^{-1/2}2 \tfrac{2}{3}=\frac{4}{3}(2\pi)^{-1/2} = 0.531923 
		\end{align*}
		
				
		\begin{align*}
			\Gamma &= \int_{0}^{\infty}\int_{0}^{2\pi} A^{m}_{n}(\rho)A^{m'}_{n'}A^{m''}_{n''}(\rho)(\rho)\rho d\rho d\varphi \\
			&= \int_{0}^{\infty}\int_{0}^{2\pi} 8\frac{n!}{(n+m)!}\frac{n'!}{(n'+m')!}\frac{n''!}{(n''+m'')!}L^{m}_{n}(x)L^{m'}_{n'}(x)L^{m''}_{n''}(x)\tfrac{1}{4}x^{(m+m'+m'')/2}e^{-\tfrac{3}{2}x}B^{i}_{m}(\varphi)B^{i'}_{m'}(\varphi)B^{i''}_{m''}(\varphi) dx d\varphi \\
			& \approx \sum_{j=1,k=1}^{N,M}\hat{w}_M w_N^{(m+m'+m'')\tfrac{1}{2}}  2\frac{n'!}{(n'+m')!}\frac{n''!}{(n''+m'')!}L^{m}_{n}(x_j)L^{m'}_{n'}(x_j)L^{m''}_{n''}(x_j)B^{i}_{m}(\varphi_k)B^{i'}_{m'}(\varphi_k)B^{i''}_{m''}(\varphi_k)\pi e^{-x_j/2}
		\end{align*}
		
	\subsection{Cuadratura para las proyecciones}
	
		Sea $f(\rho,\varphi)$, siguiendo el mismo procedimiento que con la cuadratura de los acoples no lineales ($\Gamma$) pero observando que ahora $\alpha=m/2$.
		
		\begin{align*}
			\langle f,A_{pmi} \rangle &= \int_{0}^{\infty}\int_{0}^{2\pi} f(\rho,\varphi) A^{m}_{n}(\rho)\rho d\rho d\varphi\\
			&= \int_{0}^{\infty}\int_{0}^{2\pi} 2 (\frac{n!}{(n+m)!})^{1/2} f(\sqrt{x/2},\varphi) L^{m}_{n}(x)\tfrac{1}{4}x^{m/2}e^{-x/2}B^{i}_{m}(\varphi) dx d\varphi\\
			&\approx \sum_{j=1,k=1}^{N,M}\hat{w}_M w_N^{m/2}  \frac{1}{2}(\frac{n!}{(n+m)!})^{1/2} f(\sqrt{x_j/2},\varphi_k) L^{m}_{n}(x_j)B^{i}_{m}(\varphi_k)\pi e^{x_j/2}
		\end{align*}	
		
		Dado que en este caso las funciones que se integran no forman parte de la familia de funciones usadas para la cuadratura, se espera un grado de convergencia mas lento.
		Por esta razón se realizo un estudio de convergencia para diferentes tanto en la parte radial como en la angular.
		
		
		\begin{figure}[h]
			\includegraphics[width= 0.5\linewidth]{../../2d maxw galerkin/2d_maxwellblock_gallerkin/Python/Imagenes/cv_2d_proy_pump (n= 1).png}
			\includegraphics[width= 0.5\linewidth]{../../2d maxw galerkin/2d_maxwellblock_gallerkin/Python/Imagenes/cv_proy_pump (n= 1).png}
			\caption{ Izquierda: Mapa de resultados obtenidos al variar el orden de los parametros de integracion. 
				Derecha: Resultados obtenidos al aumentar el orden de integración radial para un orden fijo en la parte angular }
		\end{figure}
		
		
		\begin{figure}[h]
			\includegraphics[width= 0.5\linewidth]{../../2d maxw galerkin/2d_maxwellblock_gallerkin/Python/Imagenes/cv_2d_proy_pump (n= 25).png}
			\includegraphics[width= 0.5\linewidth]{../../2d maxw galerkin/2d_maxwellblock_gallerkin/Python/Imagenes/cv_proy_pump (n= 25).png}
			\caption{Izquierda: Mapa de resultados obtenidos al variar el orden de los parametros de integración. 
				Derecha: Resultados obtenidos al aumentar el orden de integración radial para un orden fijo en la parte angular }
		\end{figure}
		
		Las funciones utilizadas no tiene una dependencia angular, los que explicaría porque las variaciones al cambiar el orden son despreciables.	
	
	
		
	\subsection{Aproximación del bombeo}
		
		Si el bombeo es definido como:
		\[\chi(\rho)=5-\tfrac{1}{2}\tanh(5(\rho-\rho_0)) \]
		con $\rho_0=1$
		
		\textcolor{red}{cambiar los $\chi$ por $f$?? u otra cosa.}
		
		Dada la simetría angular, la proyección del bombeo en la base $A_{nmi}$ unicamente es no nula cuando $m=0$.
		Por lo tanto 
		\[\chi(\rho)=\sum_{nmi}\langle  \chi | A_{nmi}\rangle A_{nmi}=\sum_{n=0}^{\infty} \langle  \chi | A_{n01}\rangle A_{n01}=\chi_n(\rho)\]
		\textcolor{red}{escribir esto para cualquier funcion con simetria en $\rho$}
		A continuación se muestra el perfil del bombeo (en azul) y la suma de las proyecciones hasta orden $n$ (en rojo).
		
		\begin{minipage}{0.33\textwidth}
			\begin{center}
				\includegraphics[width= \linewidth]{../../2d maxw galerkin/2d_maxwellblock_gallerkin/Python/Imagenes/Pump_aprox (n= 1).png}
				\includegraphics[width= \linewidth]{../../2d maxw galerkin/2d_maxwellblock_gallerkin/Python/Imagenes/Pump_aprox (n= 6).png}
			\end{center}
		\end{minipage}
		\begin{minipage}{0.33\textwidth}
			\begin{center}
				\includegraphics[width= \linewidth]{../../2d maxw galerkin/2d_maxwellblock_gallerkin/Python/Imagenes/Pump_aprox (n= 35).png}
				\includegraphics[width= \linewidth]{../../2d maxw galerkin/2d_maxwellblock_gallerkin/Python/Imagenes/Pump_aprox (n= 35)-1.png}
			\end{center}
		\end{minipage}
		\begin{minipage}{0.33\textwidth}
			\begin{center}
				\includegraphics[width= \linewidth]{../../2d maxw galerkin/2d_maxwellblock_gallerkin/Python/Imagenes/Pump_aprox (n= 20).png}
				\includegraphics[width= \linewidth]{../../2d maxw galerkin/2d_maxwellblock_gallerkin/Python/Imagenes/Pump_aprox (n= 40).png}
			\end{center}
		\end{minipage}
		
		%		\begin{center}
		%			\includegraphics[width= 0.6\linewidth]{../../2d maxw galerkin/2d_maxwellblock_gallerkin/gain-aprox.png}
		%		\end{center}
		
		Lo que indica para nuestro problema debemos utilizar $n>40$ para obtener resultados coherentes.	
		
		Comparación entre las curvas propuestas para el bombeo y para las perdidas.
		
		%%		\begin{center}
		%%			\includegraphics[width= 0.5\linewidth]{../../2d maxw galerkin/Fortran/2d_maxwellblock_gallerkin/Fortran/gain-loss.png}
		%%		\end{center}
%		
%		\begin{figure}[h]
%			\includegraphics[width= 0.5\linewidth]{../../2d maxw galerkin/2d_maxwellblock_gallerkin/Python/Imagenes/cv_2d_proy_pump (n= 1).png}
%			\includegraphics[width= 0.5\linewidth]{../../2d maxw galerkin/2d_maxwellblock_gallerkin/Python/Imagenes/cv_proy_pump (n= 1).png}
%			\caption{ }
%		\end{figure}
%		
%	
		
%		\begin{figure}[h]
%			\includegraphics[width= 0.5\linewidth]{../Python/Imagenes especiales para la tesis/chequeo_1khz}
%			\caption{Chequeo del reescaleo. $\cos(wt)$ con $w=6,28$kHz, o $\nu = 1$kHz. $\hat{w}=0,00000628$}
%		\end{figure}