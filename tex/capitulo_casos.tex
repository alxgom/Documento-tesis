\section{casos particulares}

	
	\begin{center}
		\includegraphics[width= 0.6\linewidth]{../Python/integ directa/Results/2016_7_27-13.38.21-E_intensitys.png}
		\captionof{figure}{Evolución del sistema sin modulación a partir de la condición inicial asimétrica: $E=(1+i1,1+i1)$ ,$ P=(300+i300,300+i300)$, $D=6000$}
	\end{center}


	
	\begin{center}
		\includegraphics[width= 0.6\linewidth]{../Python/integ directa/Results/2016_7_27-13.13.55-E_intensitys.png}
		\captionof{figure}{Evolución del sistema sin modulación a partir de la condición inicial asimétrica:$E=(1+i1,1+i1)$ ,$ P=(300+i300,300+i300)$, $D=6000$}
	\end{center}

	\subsection{Visualización de casos particulares }
	
	Para poder visualizar mejor le periodicidad de las soluciones obtenidas y obtener mas información de cada gráfico, se realizaron visualizaciones en las cuales se representa en un mapa de colores divergente los valores que toma la modulación .
	
	En la figura \ref{fig: ej mod} se muestra un gráfico de la modulación con este método de visualización, para demostrar que según el mapa de colores utilizado, el color rojo indica cuando la modulación toma valores positivos y en azul se muestra cuando la modulación toma valores negativos.
	
	
	\begin{center}
		\includegraphics[width= 0.6\linewidth]{../Python/Imagenes especiales para la tesis/map_cos.png}
	\end{center}
	
	La ventaja de este tipo de visualizaciones es que me permite ver de manera mas simple la modulación del sistema, y por lo tanto agregarle una dimensión mas a los gráficos.
	
	Comparando dos gráficos de la misma serie temporal, a la izquierda se ve en azul la serie temporal del modulo del campo eléctrico en función del tiempo(eje de abajo) o de la cantidad de periodos de la modulación (eje de arriba) y en verde se muestra la modulación . A la derecha se muestra el mismo resultado con el método de visualización, donde en azul se puede ver cuando la modulación es negativa y en rojo cuando es positiva.
	
	\begin{minipage}{0.5\textwidth}
		
		\centering
		\includegraphics[width= 1\linewidth]{../Python/integ directa/Results/2016_5_28-11.50.45-comparison.png}
		\capionof{figure}{Ejemplo del método de visualización de la fase para la modulación.}
		\label{fig: ej mod}
		
	\end{minipage}
	\begin{minipage}{0.5\textwidth}
		
		\centering
		\includegraphics[width= 1\linewidth]{../Python/integ directa/Results/2016_5_28-11.50.48-color_comparison.png}
		%\caption{Set joke}
		%	\label{fig:erise}
		
	\end{minipage}
	
	Visualización de la dinámica del modulo del campo eléctrico . A la izquierda se muestra la serie temporal, mientras que a la derecha se puede ver la evolución en la proyección del espacio formado por el modulo del campo eléctrico y la población del láser .
	
	\begin{minipage}{0.5\textwidth}
		
		\centering
		\includegraphics[width= 1\linewidth]{../Python/integ directa/Results/2016_5_31-13.32.21-color_comparison.png}
		%\caption{Set joke}
		%	\label{fig:erise}
		
	\end{minipage}
	\begin{minipage}{0.5\textwidth}
		
		\centering
		\includegraphics[width= 1\linewidth]{../Python/integ directa/Results/2016_5_31-13.35.10-color_phase_E_vs_pop.png}
		%\caption{Set joke}
		%	\label{fig:erise}
		
	\end{minipage}
	
	
	Para los casos en los que la dinámica presentaba un decaimiento en los campos en $x$, se realizaron visualizaciones del las componentes del campo complejo en $y$, la población y , en color, la modulación.
	
	Las razones para esta visualización es que si los campos en $x$ decaen, puedo eliminarlos del sistema de ecuaciones. Por otro lado, para todo los casos estudiados , incluso cuando ambos campos tiene dinámicas periódicas, siempre se verifico que la polarización sigue aproximadamente el mismo comportamiento que la intensidad. Lo cual indicaría que es podría ser posible realizar una eliminación adiabatica de la población y así eliminar otra dimensión mas del problema. 
	\begin{center}
		\includegraphics[width= 0.4\linewidth]{../Python/integ directa/Results/2016_5_28-23.57.19-E_vs_P.png}
	\end{center}
	\textcolor{red}{cambiar figura, poner las otras.}
	Comparacion del comportamiento del modulo de la población y el modulo del campo eléctrico.
	
	%	\textcolor{red}{Podría hacer uno con Re(Py) y Re(Ey), y con las partes imaginarias, para no usar el modulo, y probar que incluso sigue la dinámica rápida los campos eléctricos.}
	
	
	A continuación se pueden ver ejemplos para trayectorias de periodo dos
	
	\begin{minipage}{0.5\textwidth}
		
		\centering
		\includegraphics[width= 1\linewidth]{../Python/integ directa/Results/2016_5_28-5.0.56-color_comparison.png}
		%\caption{Set joke}
		%	\label{fig:erise}
		
	\end{minipage}
	\begin{minipage}{0.5\textwidth}
		
		\centering
		\includegraphics[width= 1\linewidth]{../Python/integ directa/Results/2016_5_28-4.50.26-color_space.png}
		%\caption{Set joke}
		%	\label{fig:erise}
		
	\end{minipage}
	
	Para una trayectoria de periodo 1
	
	\begin{minipage}{0.5\textwidth}
		
		\centering
		\includegraphics[width= 1\linewidth]{../Python/integ directa/Results/2016_5_28-11.50.48-color_comparison.png}
		%\caption{Set joke}
		%	\label{fig:erise}
		
	\end{minipage}
	\begin{minipage}{0.5\textwidth}
		
		\centering
		\includegraphics[width= 1\linewidth]{../Python/integ directa/Results/2016_5_28-11.31.0-color_space.png}
		%\caption{Set joke}
		%	\label{fig:erise}
		
	\end{minipage}
	
	Se observa que las trayectorias no son cerradas, esto podría indicar que la fase del campo complejo no esta acoplada  a la modulación.
	
	\subsection{Fase física}
	
	\begin{minipage}{0.5\textwidth}
		
		\centering
		\includegraphics[width= 1\linewidth]{../Python/integ directa/Results/2016_6_2-2.48.0-E_intensitys.png}
		\includegraphics[width= 1\linewidth]{../Python/integ directa/Results/2016_6_2-2.28.54-color_comparison}
		%\caption{Set joke}
		%	\label{fig:erise}
		
	\end{minipage}
	\begin{minipage}{0.5\textwidth}
		
		\centering
		\includegraphics[width= 1\linewidth]{../Python/integ directa/Results/2016_6_2-2.52.36-color_physical_pol.png}
		%\caption{Set joke}
		%	\label{fig:erise}
		
	\end{minipage}
	
	\begin{minipage}{0.5\textwidth}
		
		\centering
		\includegraphics[width= 1\linewidth]{../Python/integ directa/Results/2016_6_10-3.0.44-E_intensitys.png}
		%\caption{Set joke}
		%	\label{fig:erise}
		\includegraphics[width= 1\linewidth]{../Python/integ directa/Results/2016_6_10-3.1.33-color_comparison.png}
		
	\end{minipage}
	\begin{minipage}{0.5\textwidth}
		
		\centering
		\includegraphics[width= 1\linewidth]{../Python/integ directa/Results/2016_6_10-3.9.17-color_physical_pol.png}
		%\caption{Set joke}
		%	\label{fig:erise}
		
	\end{minipage}