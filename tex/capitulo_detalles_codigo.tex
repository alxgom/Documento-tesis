\section{Detalles del código}

Para realizar las integraciones del código se utilizo la librería \texttt{scypi.odeint}.

Como la integración se realiza en la  escala temporal de $\tau$ los resultados deben ser reescaleados .

Para reescalear los tiempos se multiplica a los mismos por $\gamma_{\bot}$, por ultimo se multiplica a esta valor por $10^{-6}$ para pasar la escala a microsegundos.
Para la frecuencia utilizada, como la misma aparece en $\cos(\hat{w}\tau)$, $\hat{w}=w \gamma_{\bot}$. Por lo tanto para reescalear $\hat{w}$ en $w$ se $w=\frac{\hat{w}}{\gamma_{\bot}}$. Por ultimo para pasar este valor a kHz, se lo multiplica por  $10^{-3}$.

\begin{figure}[h]
	\includegraphics[width= 0.5\linewidth]{../Python/Imagenes especiales para la tesis/chequeo_1khz}
	\caption{Chequeo del reescaleo. $\cos(wt)$ con $w=6,28$kHz, o $\nu = 1$kHz. $\hat{w}=0,00000628$}
\end{figure}

Chequeo del reescaleo. $\cos(wt)$ con $w=6,28$kHz, o $\nu = 1$kHz. $\hat{w}=0,00000628$ .

	\subsection{Código utilizado para los barridos.}
	
	\begin{lstlisting}

"""
Integrates the maxwell-block equations with a modulation on the polarization phase.
Swypes the value of the frequency modulation and output files needed to plot an
stroboscopic map.

Modification of swype_w_max so that the new initial condition for each step comes from the last maximum value of the previous iteration.
This might me usefull to better tack trayectories from others wich migh share a basin in some part of the phase space, but not where the maximum is reached.

"""

import numpy as np
from scipy.integrate import odeint
from scipy.signal import argrelextrema
from time import localtime
import time as timing
import shutil

pi=np.pi

def editline(file,n_line,text):
	with open(file) as infile:
	lines = infile.readlines()
	lines[n_line] = text+' \n'
	with open(file, 'w') as outfile:
	outfile.writelines(lines)  

def mb(y, t,k,mu,Dphi0,d,g,D0,m,wf):
	""" y[0],y[1] campo electrico en x. y[2],y[3] campo electrico en y, y[4],y[5]  polarizacion en x, y[6],y[7]  polarizacion en y, y[8] poblacion. """
	dfxr=-k*y[0]+mu*y[4]
	dfxi=-k*y[1]+mu*y[5]
	dfyr=-k*y[2]+mu*y[6]-y[3]*(Dphi0+m*np.cos(wf*t))
	dfyi=-k*y[3]+mu*y[7]+y[2]*(Dphi0+m*np.cos(wf*t))
	drxr=-(1*y[4]-d*y[5])+y[0]*y[8]
	drxi=-(1*y[5]+d*y[4])+y[1]*y[8]
	dryr=-(1*y[6]-d*y[7])+y[2]*y[8]
	dryi=-(1*y[7]+d*y[6])+y[3]*y[8]
	ddelta=-g*(y[8]-D0+(y[0]*y[4]+y[1]*y[5]+y[2]*y[6]+y[3]*y[7]))
	return [dfxr,dfxi,dfyr,dfyi,drxr,drxi,dryr,dryi,ddelta]

#%%
#Read parameters from 'Params.txt', set.
txtfile=open("Params.txt")
tempvar=txtfile.readlines()

'''parameters for normalization'''
a= float(tempvar[1].split()[1])
gperp= float(tempvar[2].split()[1])  #gamma perpendicular, loss rate
scale=1*(10.**6)/gperp #scale to micro seconds
wscale=1000*gperp/(10.**6)#scale frequency to khz

'''parameters for the equation'''
kr= float(tempvar[3].split()[1])
k=kr/gperp #normalized loss rate
mu= float(tempvar[4].split()[1])
Dphi0= float(tempvar[5].split()[1]) #phase shift [-pi,pi]
d= float(tempvar[6].split()[1])#detuning
gr= float(tempvar[7].split()[1])
g=gr/gperp #*((2*pi)**2) #sigma parallel, normalized loss rate
D0=a*k/mu #Poblation
m= float(tempvar[8].split()[1]) #modulation amplitud [0,1]
wf= float(tempvar[9].split()[1])
print 'Params: ', a,gperp,kr,mu,Dphi0,d,gr,m,wf

'''parameters to compare with the results'''
w_res=np.sqrt(k*g*((D0*mu/k)-1.))*wscale #resonance frequency
a=D0*mu/k
w=np.sqrt(k*g*(a-1.)-(g*g*a*a)/4)*wscale #Relaxation oscilations frequency
wf_real=wf*wscale

'''Parameters for swype'''
par_max= float(tempvar[11].split()[1])
par_min= float(tempvar[12].split()[1])
h= float(tempvar[13].split()[1])
direction= tempvar[16].split()[1]

txtfile.close()

txtfile=open("Status.txt",'r')
tempvar=txtfile.readlines()
stat= int(tempvar[0].split()[1])
print 'Status=', stat
txtfile.close()#deberia cerrarlo aca?    

#%%        
def swipe(m,k,mu,Dphi0,d,g,D0,wf,par_max,par_min,h,direction):
	'''Swype parameters'''
	
	nintime_periods=110
	nperiods=50
	
	if direction != 'increasing' and direction != 'decreasing' :
		'''Swype direction bad defined '''
		print 'Error!. Swype direction Bad defined'
		return
	
	print 'Program: swype_w_max'
	print 'total number of steps: %s' %((par_max-par_min)/h)
	print 'direction: ' , direction
	
	if stat==0:    
		'''User defined initial condition'''
		dim=9        
		yinit=np.empty(dim)
		txtfile=open("Params.txt",'r')
		tempvar=txtfile.readlines()
		for j in range(dim):
			yinit[j]=float(tempvar[22].split()[-dim+j])
		timeinit=float(tempvar[19].split()[1])
		txtfile.close()
		
		if direction=='increasing':
			wf=par_min   
		elif direction=='decreasing':
			wf=par_max
		
		binwrite=open('w_max.in','wb')#clear w_max and strobo
		binwrite.close()
		binwrite=open('max_peak.in','wb')
		binwrite.close()
		binwrite=open('benchmark.in','wb')# to save runtimes
		binwrite.close()
		binwrite=open('Initial_conditions.txt','w')# to save runtimes
		binwrite.close()
		
		editline('Params.txt',9,'wf: %f' %wf)#update next wf to file
		editline('Status.txt',0,'Stat: 1') #set status to continue
	
	elif stat==1:  
		yinit=np.fromfile('yinitials.in',dtype=np.float64)
		timeinit=np.fromfile('tinitials.in',dtype=np.float64)
	
	elif stat==2: # if stat =2 rerun from manual usr input.
		txtfile=open("Status.txt",'r')
		tempvar=txtfile.readlines()
		old_w=float(tempvar[3].split()[1])
		old_time=float(tempvar[3].split()[4])
		dim=9
		old_y=np.empty(9)
		for j in range(dim):
			old_y[j]=float(tempvar[3].split()[-dim+j])
		np.ndarray.tofile(old_y,'yinitials.in')
		np.ndarray.tofile(np.array(old_time),'tinitials.in')#no necesito abrir el archivo con
		yinit=old_y        
		timeinit=np.array([old_time])               
		wf=old_w
		editline('Params.txt',9,'wf: '+repr(wf))#update next m to file               
		editline('Status.txt',0,'Stat: 1')      
	
	'''swipe''' 
	while wf<(par_max+h):
		if wf<par_min:
			return
	
		tin=timing.time()
		
		#intime=500.*11.3*10**(-6)*gperp #integration time FOR TRANSITORY  
		intime=nintime_periods*2*pi/wf
		keeptime=nperiods*2*pi/wf
		
		print 'integrating wf= '+repr(wf), 
		
		newline=open('Initial_conditions.txt','a')
		newline.write('wf= '+repr(wf)+' , tinit= '+repr(timeinit)+' , yinit= '+repr(yinit[0])+' '+repr(yinit[1])+' '+repr(yinit[2])+' '+repr(yinit[3])+' '+repr(yinit[4])+' '+repr(yinit[5])+' '+repr(yinit[6])+' '+repr(yinit[7])+' '+repr(yinit[8])+' \n' )          
		newline.close()
		
		'''Transitory integration'''
		trans_step=int(intime/20.)
		time_trans = np.linspace(timeinit ,intime+timeinit , trans_step) #cambiar por linspace
		y_trans = odeint(mb, yinit,time_trans,args=(k,mu,Dphi0,d,g,D0,m,wf))
		timeinit=time_trans[-1]
		
		
		'''integration'''
		time_step=int(keeptime/1.)
		time = np.linspace(timeinit ,keeptime+timeinit , time_step) #cambiar por linspace
		y = odeint(mb, y_trans[-1], time,args=(k,mu,Dphi0,d,g,D0,m,wf))
		
		'''set intensitys'''
		#        intensity_ex=np.sqrt(y[:,0]**2+y[:,1]**2)
		#        intensity_ey=np.sqrt(y[:,2]**2+y[:,3]**2)
		intensity=np.sqrt(y[:,0]**2+y[:,1]**2+y[:,2]**2+y[:,3]**2)
		
		'''map'''
		extrema=argrelextrema(intensity, np.greater)[0]
		peak_max=intensity[extrema]#intensity strobo. puedo scar el set
		w_peaks=wf*np.ones_like(peak_max)#vector or m, the same lenght as peak_max     
		
		if direction=='increasing':
			wf=wf+h   #ste next wf
			timeinit=time[extrema[-1]]*(1-h/wf)#New initial time is the last maximum time
		elif direction=='decreasing':
			wf=wf-h
			timeinit=time[extrema[-1]]*(1+h/wf)#New initial time is the last maximum time
		
		yinit=y[extrema[-1]]
		
		'''save'''
		editline('Params.txt',9,'wf: '+repr(wf))#update next wf to file
		
		binwrite=open('w_max.in','ab')
		w_peaks.tofile(binwrite)
		binwrite.close()
		binwrite=open('max_peak.in','ab')
		peak_max.tofile(binwrite)
		binwrite.close()
		
		np.ndarray.tofile(yinit,'yinitials.in')
		np.ndarray.tofile(np.array(timeinit),'tinitials.in')
		
		runtime=timing.time()-tin
		binwrite=open('benchmark.in','ab')
		np.array([wf+h,runtime]).tofile(binwrite)
		binwrite.close() 
		print 'runtime: %.2f s' %(runtime)
		
	return 

#%%    
'''run main program'''
swipe(m,k,mu,Dphi0,d,g,D0,wf,par_max,par_min,h,direction)    
#%%
editline('Status.txt',0,'Stat: 0') #set status to continue

#%%
'''write the results to a subfolder ../Results  (so i can plot them again if needed)'''

binwrite=open('Results/%d_%d_%d-%d.%d.%d-w_max.in' % tuple(list(localtime()[0:6])),'wb') 
w_peks=np.fromfile('w_max.in',dtype=np.float64)
w_peks.tofile(binwrite)
binwrite.close()        
binwrite=open('Results/%d_%d_%d-%d.%d.%d-max_peak.in'  %localtime()[0:6] ,'wb')
peak_max=np.fromfile('max_peak.in',dtype=np.float64)
peak_max.tofile(binwrite)
binwrite.close()

shutil.copyfile('Initial_conditions.txt','Results/'+'%d_%d_%d-%d.%d.%d-Initial_conditions.txt' %localtime()[0:6])
shutil.copyfile('Params.txt','Results/'+'%d_%d_%d-%d.%d.%d-params.txt' %localtime()[0:6])

	\end{lstlisting}
%	Para realizar las integraciones del código se utilizo la librería \texttt{spypi.odeint}.
	
%	Como la integración se realiza en la  escala temporal de $\tau$ los resultados deben ser reescaleados .
	
%	Para reescalear los tiempos se multiplica a los mismos por $\gamma_{\bot}$, por ultimo se multiplica a esta valor por $10^{-6}$ para pasar la escala a microsegundos.
%	Para la frecuencia utilizada, como la misma aparece en $\cos(\hat{w}\tau)$, $\hat{w}=w \gamma_{bot}$. Por lo tanto para reescalear $\hat{w}$ en $w$ se $w=\frac{\hat{w}}{\gamma_{\bot}}$. Por ultimo para pasar este valor a kHz, se lo multiplica por  $10^{-3}$.
%	
%%	\begin{figure}
%		\includegraphics[width= 0.4\linewidth]{../Python/Imagenes especiales para la tesis/chequeo_1khz}
%%		\caption{Chequeo del reescaleo. $\cos(wt)$ con $w=6,28$kHz, o $\nu = 1$kHz. $\hat{w}=0,00000628$}
%%	\end{figure}
%	
%	Chequeo del reescaleo. $\cos(wt)$ con $w=6,28$kHz, o $\nu = 1$kHz. $\hat{w}=0,00000628$ .