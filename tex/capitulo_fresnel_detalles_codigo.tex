\section{Detalles del código 2D}

	Para realizar la simulaciones se realizo primero un código en Python, y luego  en Fortran.
	
	\begin{figure}[h]
		\includegraphics[width= 0.6\linewidth]{../../2d maxw galerkin/2d_maxwellblock_gallerkin/Fortran/functions/Test_functions/figure_2.png}
		\caption{Comparación entre los polinomios de Laguerre definidos en Python y en Fortran . }
	\end{figure}
	\begin{figure}[h]
		\includegraphics[width= 0.6\linewidth]{../../2d maxw galerkin/2d_maxwellblock_gallerkin/Fortran/functions/Test_functions/figure_4.png}
		\caption{Comparacion de función radial definida en Python y en Fortran }
	\end{figure}
	\begin{figure}[h]
		\includegraphics[width= 0.6\linewidth]{../../2d maxw galerkin/2d_maxwellblock_gallerkin/Fortran/functions/Test_functions/figure_5-1.png}
		\caption{Comparacion de las funciones de la base ortonormal definida en Python y en Fortran }
	\end{figure}
	\begin{figure}[h]
		\includegraphics[width= 0.6\linewidth]{../../2d maxw galerkin/2d_maxwellblock_gallerkin/Fortran/functions/Test_functions/figure_6.png}
		\caption{Comparacion de la parte radial de las  funciones de la base ortonormal definida en Python y en Fortran }
	\end{figure}
	\begin{figure}[htp]
		\includegraphics[width= 0.6\linewidth]{../../2d maxw galerkin/2d_maxwellblock_gallerkin/Fortran/functions/Test_functions/figure_7-1.png}
		\caption{Comparacion de las funciones de la base ortonormal definida en Python y en Fortran }
	\end{figure}

					
	

