\section{Ruptura de simetria y decaimiento de $E_x$}

	\subsection{Intensidad y fase }
	
	Intensidades:\\
	
	Para la situaciones en las que decae el campo en la direccion $X$, si no se tiene en cuenta toda la dinamica en esa direccion, se intenta reescribir el termino de en $y$ para la intensidad y la fase.
	
	Definiendo $I_y=Re(E_y)^2+Im(E_y)^2$, y recordando que 
	\[
	\begin{cases}
	\partial_{\tau} Re(E_y)=-k Re(E_y) + \mu Re(P_y) -(\Delta \phi_0 + m.cos(w_{mod}.\tau)).Im(E_y) \\
	\partial_{\tau} Im(E_y)=-k Im(E_y) + \mu Im(P_y) + (\Delta \phi_0 + m.cos(w_{mod}.\tau)).Re(E_y) \\
	\end{cases}
	\]
	
	$\partial_{\tau}I_y=2 Re(E_y)\partial_{\tau}Re(E_y)+2 Im(E_y)\partial_{\tau}Im(E_y)=E_y^*\partial_{\tau}E_y+E_y\partial_{\tau}E_y^*$
	
	por lo tanto queda que 
	
	$\partial_{\tau}I_y=2 [Re(E_y)(-k Re(E_y) + \mu Re(P_y) - \Delta \phi.Im(E_y))+ Im(E_y)(-k Im(E_y) + \mu Im(P_y) + \Delta \phi .Re(E_y))$] \\
	
	con lo cual se anulan los términos que tiene la modulación de manera explicita. Reescribiendo un poco la ecuación, se obtiene
	
	\begin{equation}
	\partial_{\tau}I_y=2[-k I_y +\tfrac{1}{2}\mu[E^*_yP_y+E_yP^*_y]   ] 		
	\end{equation}
	
	
	Dado que la modulación se anula, el mismo procedimiento se puede realizar para la intensidad en x y la intensidad total.
	
	\begin{align}
	\partial_{\tau}I_x &= 2[-k I_x +\tfrac{1}{2}\mu[E^*_xP_x+E_xP^*_x] ]  \\
	\partial_{\tau}I &= 2[-k I +\tfrac{1}{2}\mu[E^*_xP_x+E_xP^*_x+E^*_yP_y+E_yP^*_y]]
	\end{align}
	
	Fase:\\
	
	Siendo que para un numero complejo $Z$, $\psi=\arctan(\frac{Im(z)}{Re(z)})$
	
	\[  \psi_y= \arctan(\frac{Im(E_y)}{Re(E_y)}) \]
	
	por  lo tanto , notando $Im(E_y)=E_{yi}$, $Re(E_y)=E_{yr}$ y $\Delta \phi=\Delta \phi_0 + m.cos(w_{mod}.\tau) $ :
	
	\begin{align*}
	\partial_{\tau}\psi_y  &= \frac{1}{ 1+\frac{ E^2_{yi} }{ E^2_{yr} } }\frac{ E_{yr} \partial_{\tau}( E_{yi} ) - E_{yi} \partial_{\tau}( E_{yr} ) }{ E^2_{yr} } \\ 
	&= \frac{ E_{yr} ( -kE_{yi}+\mu P_{yi}+\Delta \phi E_{yr} ) -  E_{yi} ( -kE_{yr}+\mu P_{yr}-\Delta \phi E_{yi} ) }{ I_y } \\
	&= \frac{ \mu ( P_{yi} E_{yr} - P_{yr} E_{yi} ) }{ I_y } + \Delta \phi	
	\end{align}
	
	Llamativamente, queda de manera explicita la dependencia con la modulación.
	
	Haciendo lo mismo en $x$, las ecuaciones se pueden reescribir como :
	\begin{equation}
	\begin{cases}
	\partial_{\tau}I_y=2[-k I_y +\tfrac{1}{2}\mu[E^*_yP_y+E_yP^*_y]   ] \\
	\partial_{\tau}I_x=2[-k I_x +\tfrac{1}{2}\mu[E^*_xP_x+E_xP^*_x]   ]	\\	
	\partial_{\tau}\psi_y  = \frac{ \mu ( P_{yi} E_{yr} - P_{yr} E_{yi} ) }{ I_y } + \Delta \phi \\
	\partial_{\tau}\psi_x  = \frac{\mu(P_{xi}E_{xr}-P_{xr}E_{xi})}{I_x}
	
	\end{cases}
	\end{equation}
	
	
	%		\subsubsection{Eliminación adiabatica}
	%		
	%		Si se supone que la polarizacion es proporcional a los campos electricos respectivos, osea $P_{x,y} = \alpha E_{x,y}$, entonces 
	%		
	%		\[ 
	%			\begin{cases}
	%				\partial_{\tau}I^2_y=2[-k I^2_y +\tfrac{1}{2}\mu \alpha I^2_y ] \\  
	%				\partial_{\tau}I^2_x=2[-k I^2_x +\tfrac{1}{2}\mu \alpha I^2_x ]\\
	%				\partial_{\tau}\psi_y  = \Delta \phi\\
	%				\partial_{\tau}\psi_x  = 0
	%			\end{cases}
	%		\]
	
	
	Finalmente, para la polarización del haz.
	
	$\Psi=\arctan(\frac{E_{yr} }{E_{xr} })$
	
	\begin{equation}
	\partial_{\tau}\Psi  = \frac{\mu(P_{yr}E_{xr}-P_{xr}E_{yr})-\Delta \phi E_{xr}E_{yr}}{E^2_{yr}+E^2_{xr}}
	\end{equation}
	
	
	
	%		por ultimo, como $I=\rho^2$, entonces $\partial_{\tau}I=2 \partial_{\tau}\rho$
	%		
	%		\[
	%		\begin{cases}
	%		\partial_{\tau} I_x=2[-k \rho_x + \alpha \mu\rho_x D] \\
	%		\partial_{\tau} \varphi_x=-\alpha \mu \tilde{\delta} D +\tilde{\delta w}\
	%		\partial_{\tau} I_y=2[-k \rho_y + \alpha \mu\rho_y D] \\
	%		\partial_{\tau} \varphi_y=-\alpha \mu \tilde{\delta} D +\tilde{\delta w}+\Delta \phi \\
	%		\partial_{\tau} D=-\gamma_{||}(D-D_0+\alpha D I) \\
	%		\end{cases}
	%		\]
	%		