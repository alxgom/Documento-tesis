\section{Multiestabilidad}
\textcolor{red}{ahora estoy pasando estos resultados a donde están los mapas. Tengo que modificarlo.}

Como ejemplo de un caso claro de multiestabilidad, a continuación se muestran dos evoluciones temporales del modulo del campo eléctrico comparadas con la modulación. 
En ambos casos se utilizaron los mismos parámetros, pero distintas condiciones iniciales.

	\subsection{$m=0.02$, $w=237.3$Khz}
		\begin{minipage}{0.5\textwidth}
			
			\centering
			\includegraphics[width= 1\linewidth]{../Python/integ directa/Results/2016_5_17-15.33.6-comparison.png}
			%\caption{Set joke}
		%	\label{fig:erise}
			
		\end{minipage}
		\begin{minipage}{0.5\textwidth}
			
			\centering
			\includegraphics[width= 1\linewidth]{../Python/integ directa/Results/2016_5_17-15.33.8-E_vs_population.png}
			%\caption{Set joke}
		%	\label{fig:erise}
			
		\end{minipage}
		
		
		
		\begin{minipage}{0.5\textwidth}
			
			\centering
			\includegraphics[width= 1\linewidth]{../Python/integ directa/Results/2016_5_17-15.18.58-comparison.png}
		
		\end{minipage}
		\begin{minipage}{0.5\textwidth}
			
			\centering
			\includegraphics[width= 1\linewidth]{../Python/integ directa/Results/2016_5_17-15.19.0-E_vs_population.png}
			%\caption{Set joke}
		%	\label{fig:erise}
			
		\end{minipage}
	
	
	\subsection{$m=0.02$, $w=100$Khz}
		
		Se muestran tres órbitas estables para $m=0.02$, $w=100$Khz.
		
		\begin{minipage}{0.33\textwidth}
		\centering
		\includegraphics[width= 1\linewidth]{../Python/integ directa/Results/multiestability/2016_6_1-22.53.21-color_phase_E_vs_pop.png}
		%\caption{Set joke}
		%	\label{fig:erise}
		
		\end{minipage}
		\begin{minipage}{0.33\textwidth}
		
		\centering
		\includegraphics[width= 1\linewidth]{../Python/integ directa/Results/multiestability/2016_6_1-23.0.5-color_phase_E_vs_pop.png}
		
		\end{minipage}
		\begin{minipage}{0.33\textwidth}
		
		\centering
		\includegraphics[width= 1\linewidth]{../Python/integ directa/Results/multiestability/2016_6_1-22.57.58-color_phase_E_vs_pop.png}
		
		\end{minipage}
		
		
		De las mimas, solo la ultima presenta dinámica no nula en ambas direcciones del campo electrico.
		
		\begin{minipage}{0.5\textwidth}
		
		\centering
		\includegraphics[width= 1\linewidth]{../Python/integ directa/Results/multiestability/2016_6_1-22.58.1-E_intensitys.png}
		%\caption{Set joke}
		%	\label{fig:erise}
		
		\end{minipage}