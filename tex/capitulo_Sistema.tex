%\documentclass[a4paper,11pt,spanish,sans]{standalone}
%\usepackage{../document_style}
%
%\begin{document}

\chapter{Modelo teorico}
	\section{El sistema:}
		
		
	\begin{figure}[htp]
		\begin{center}
			\includegraphics[width= 0.5\linewidth]{../Python/Imagenes especiales para la tesis/otro/Ring_laser.jpg}
			\caption{Ring laser comercial}
			\label{fig: real ring laser}
		\end{center}
	\end{figure}	
	
	El sistema a estudiar es un LASER en anillo (\textit{Ring LASER}), que se caracteriza por retroalimentar el medio activo en un loop o anillo.
	Usualmente en un \textit{Ring LASER}  por el medio activo viajan dos campos en dirección opuesta , con misma polarización.
	%(el campo en la otra direccion es despreciable).
	
	
	
	En nuestro caso modelamos al LASER como unidireccional.
	
	\textcolor{red}{intro ring laser del haken ,pag 47., y algo de algun paper }
	
	\begin{figure}[htc]
	\begin{center}
		
		\begin{tikzpicture}
		\tikzstyle{ground}=[fill=white,pattern=north east lines,draw=none,minimum width=0.95cm,minimum height=0.1cm]
		\tikzstyle{ground2}=[fill=white,pattern=north west lines,draw=none,minimum width=0.95cm,minimum height=0.1cm]
		\tikzstyle{mirr3}=[draw=none,minimum width=0.95cm,minimum height=0.1cm]
		\tikzstyle{medium}=[fill=orange,draw=none,minimum width=1.5cm,minimum height=0.08cm]
		\tikzstyle{faraday}=[fill=grey,draw=none,minimum width=0.5cm,minimum height=0.08cm]
		
		\coordinate (O) at (-2,-2);
		\coordinate (A) at (2,-2);
		\coordinate (B) at (2,2);
		\coordinate (C) at (-2,2);
		\coordinate (D) at (4,-2);
		
		\begin{scope}[very thick,decoration={
			markings,
			mark=at position 0.5 with {\arrow{>}}}
		] 
		\draw[ultra thick, color=red,postaction=decorate] (O)--(A);
		\draw[ultra thick, color=red,postaction=decorate] (A)--(B);
		\draw[ultra thick, color=red,postaction=decorate] (B)--(C);		
		\draw[ultra thick, color=red,postaction=decorate] (C)--(O);
		\draw[ultra thick, color=red,postaction=decorate] (A)--(D);
		\end{scope}
		
		\coordinate (z0) at (3,2);
		\coordinate (zL) at (2,3);
		
		
		%					\node at (0,-2) [rectangle,fill=orange,opacity=0.6] (1.0,0.07) {};
		%					
		\node at (0,-2) [medium,opacity=0.6] {};
		\node at (0,2) [faraday,opacity=0.6] {};
		
		\node (mirror1) at (O) [ground,anchor=north,rotate around={320:(O)}] {};
		\draw[thick] (mirror1.north west) -- (mirror1.north east) ;
		\node (mirror2) at (A) [mirr3,anchor=north,rotate around={45:(A)},label={below left:$R<1$}]  {};
		\draw[thick] (mirror2.north east) --
		(mirror2.north west) ;
		\node (mirror3) at (B) [ground,anchor=north,rotate around={135:(B)}] {};
		\draw[thick] (mirror3.north west) --
		(mirror3.north east) ;
		%							\node (labels) at (B) [label={right:$z=L$},label={above:$z=0$}] {};
		\node (mirror4) at (C) [ground2,anchor=north,rotate around={225:(C)}] {};
		\draw[thick] (mirror4.north east) --
		(mirror4.north west) ;					
		
		
		\draw (B)--(z0) ;
		\draw (B)--(zL) ;
		\tkzLabelSegment[above=1cm](B,z0){$z=0$}
		\tkzLabelSegment[right=1cm](B,zL){$z=L$}
		
		
		\end{tikzpicture}
		\caption{\textit{Ring LASER}}
		\label{fig: Ring}
	\end{center}

\end{figure}

%		
%	


		



%\end{document}	