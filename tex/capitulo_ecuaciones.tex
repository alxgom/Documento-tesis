		
				\subsection{Normalizaciones y aproximaciones}
				
				\subsubsection{Campo unidireccional}
				
				Se toma al campo como yendo en dirección positiva en $\hat{z}$ (eje longitudinal del medio), es decir que el ring laser solo funciona en una dirección.
				
				Por lo tanto  $E(r,z,t)=E_x(r,t) e^{i(k_0 z - w_0 t)} + E_y(r,t) e^{i(k_0 z - w_0 t)} $
				
				tomando \[E(r,z,t)=\frac{\hbar \sqrt{ \gamma_{\bot} \gamma_{||} }}{2\mu } \frac{1}{2} [F(r,z,t) e^{i(k_0 z - w_0 t)} + c.c.]  \]
				
				y normalizando $\tau=\gamma_{\bot} t$ , $\eta=\tfrac{z}{L}$ , $v=\frac{c}{L \gamma_{\bot}}$ por lo tanto 
				\[ \partial_t = \gamma_{\bot} \partial_{\tau} \]
				\[ \partial_z = \frac{1}{L} \partial_{\eta}  \]
				
				
				$\delta_{ac}'=\frac{w_a - w_0}{\gamma_{\bot}}$
				donde $w_a$ es la frecuencia de transición atómica, y $w_0$ la frecuencia de la longitud de onda fundamental de la cavidad.
				
				
				\subsubsection{Onda plana}
				
				Suponiendo que dentro del  medio el haz de luz es homogéneo en las direcciones transversales (.. in focus)
				
				\[ \nabla^2 \bar{E} = \nabla^2_{\bot}\bar{E} + \partial^2_z \bar{E} \approx \partial^2_z \bar{E} \]
				
				Por lo tanto 
				\begin{align*}
				\partial^2_z \bar{E} &= \frac{\hbar \sqrt{ \gamma_{\bot} \gamma_{||} }}{2\mu }\frac{1}{2} \{ [\partial^2_z F + 2i k_0 \partial_z F -k_0^2 F]e^{i(k_0 z - w_0 t)} + c.c.  \} \\  
				\partial_t \bar{E} &= \frac{\hbar \sqrt{ \gamma_{\bot} \gamma_{||} }}{2\mu }\frac{1}{2} \{ [\partial_t F - iw_0 F ] e^{i(k_0 z - w_o t)} + c.c. \}  \\
				\partial^2_t \bar{E} &= \frac{\hbar \sqrt{ \gamma_{\bot} \gamma_{||} }}{2\mu }\frac{1}{2} \{ [ \partial^2_t F -2iw_0 \partial_t F -w_0^2  F  ]e^{i(k_0-w_0t)} +c.c. \} 
				\end{align*}
				
				usando que $k_0=\frac{w_0}{c}$ , entonces $(-k_0^2+\frac{w_0^2}{c^2})F=0$
				
				\subsubsection{Variacion lenta de la amplitud}
				
				Para los LASER que es busca modelar, debido a las frecuencias en las que funcionan y a la escala temporal tipica a la que suceden las oscilaciones, se puede realizar la siguiente aproximacion:
				
				\[ \partial^2_t F\ll 2iw_0 \partial_t F
				    \]  
				
				La cual se la llama Varicion lenta de la amplitud. Parte de la validez de la ecuacion se debe a que la frecuencia del campo en los casos de interes suele estar en el rango visible.
				
				Asi como es valido realizar la aproximacion anterior, analogamente se puede realizar la misma aproximacion para la parte espacial. 
				Esto se debe en parte  a que $\bar{k}=w/c$, que en este caso como $\bar{k}=k\bar{z}$ relaciona directamente $k$ con $w$.
				
				
				 \[\partial^2_z F \ll 2ik_0\partial_z F 
				 \] 
				
				
				\[ \partial^2_{z} E - \frac{1}{c^2}\partial^2_{t}\approx  \frac{\hbar \sqrt{ \gamma_{\bot} \gamma_{||} }}{2\mu }\frac{1}{2} {[2ik_0 \partial_z F +2i\frac{w_0}{c^2}\partial_t]e^{i(k_0 z -w_0 t)}+c.c.} 
				\]
				
				De manera similar con $P$, y usando maxwell-bloch:
				
				\begin{equation}
				\begin{cases}
				\partial_{\eta}F+\frac{1}{c/(l\gamma_{\bot})}\partial_{\tau}F= -\frac{\alpha 2 \mu w_0^2}{k_0 \hbar \sqrt{ \gamma_{\bot} \gamma_{||} }} P\\
				\partial_{\tau} P=-{(1+\delta_{ac}')P+FD}\\
				\partial_{\tau}D=-\gamma_{||}\{\frac{1}{2}(F^*P+FP^*)+D-D_0\}
				\end{cases}
				\label{eq: eqs 1}
				\end{equation}  %alinear los iguales.
				
	\subsubsection{Modulacon de la fase: Modulador electro-optico.}
				
				Obs: $I=|E|^2=|E_x|^2+|E_y|^2=I_x+I_y$ porque $E_x\cdot E_y=0$. Es decir que no hay terminos de interferencia entre las dos polarizaciones. Toda la interaccion pasa por los atomos dentro del medio.
				
				Redefiniendo las constantes de la ecuacion \ref{eq: eqs 1}
				
				\begin{equation}
				\begin{cases}
				\frac{c}{l\gamma_{\bot}}\partial_{\eta}F+\partial_{\tau}F= \mu P\\
				\partial_{\tau} P=-{(1+\delta_{ac}')P+FD}\\
				\partial_{\tau}D=-\gamma_{||}\{\frac{1}{2}(F^*P+FP^*)+D-D_0\}
				\end{cases}
				\label{eq: eqs 2}
				\end{equation}  %alinear los iguales.
				
				\begin{figure}[htc]
					\begin{center}
						
						\begin{tikzpicture}
						\coordinate (O) at (-3,-0.5);
						\coordinate (A) at (0,-0.5);
						\coordinate (B) at (0,.5);
						\coordinate (C) at (-3,.5);
						\draw[fill=grey,opacity=.7] (O)--(A)--(B)--(C)--cycle;
						
						\coordinate (OO) at (0,0);
						\coordinate (D) at (4,0);
						\draw[ultra thick, color=red,->] (OO)--(D);
						
						\coordinate (F) at (4,-0.5);
						\coordinate (G) at (5,-0.5);
						\coordinate (H) at (5,.5);
						\coordinate (I) at (4,.5);
						\draw[fill=orange,pattern=north east lines,opacity=.4] (F)--(G)--(H)--(I)--cycle;
						
						\coordinate (J) at (5,0);
						\coordinate (K) at (9,0);
						\draw[ultra thick, color=red,->] (J)--(K);
						
						\tkzLabelSegment[below=2pt](F,G){Modulador Electro Optico}
						\tkzLabelSegment[above=2pt](OO,D){$I$}
						
						\end{tikzpicture}
						\caption{Modulador esquema}
						\label{fig: EOM}
					\end{center}
					
				\end{figure}
				
				
				Antes de pasar por el modulador electro óptico (EOM), $I=I_x+I_y$, luego de pasar por el modulador electro óptico $I=\alpha^2 I_x+\beta^2 I_y$
				
				\begin{equation}
				\begin{bmatrix}
				E_x\\
				E_y
				\end{bmatrix}
				=
				\begin{bmatrix}
				E_{0x}e^{i\phi_x}\\
				E_{0y}e^{i\phi_y}
				\end{bmatrix}
				e^{i(kz-wt)}
				\label{eq: EOM 1}
				\end{equation}
				
				Pensando al \textit{EOM} como una celda de Pockell que se comporta como una placa .... .
				Eligiendo los ejes del EOM, la ecuacion \ref{eq: EOM 1} como
				
				\begin{equation}
				E_{in}=
				\begin{bmatrix}
				E_{x}\\
				E_{y}e^{i\delta_y}
				\end{bmatrix}
				\label{eq: EOM 2}
				\end{equation}
				
				\begin{equation}
				E_{out}=
				\begin{bmatrix}
				E_{x}\\
				E_{y}e^{i(\delta_y + \Delta \phi)}
				\end{bmatrix}
				\label{eq: EOM 3}
				\end{equation}
				
				donde $\Delta \phi $ para el EOM esta dado por 
				
				\begin{equation}
				\Delta \phi = \Delta \phi_0 - \pi \frac{V}{V_{\pi}}
				\end{equation}
				
				donde $V_{\pi}$ es el voltaje que produce una variación en la birrefringencia equivalente a un cambio de fase de $\pi $ .
				
				\[ \Delta \phi_0 =[2\pi(n_e-n_0)\tfrac{L}{\lambda}] \]
				
				donde $n_e$ es el indice extraordinario, y $n_0$ es el indice ordinario. $ L$ es la longitud del cristal y $\lambda $ la longitud de onda en el vacio.
				
				
				\begin{equation}
				E_{out}=
				\begin{bmatrix}
				1 & 0\\
				0 & e^{-i\Delta \phi}
				\end{bmatrix}
				\begin{bmatrix}
				E_{x}\\
				E_{y}e^{i\delta_y }
				\end{bmatrix}
				\label{eq: EOM 4}
				\end{equation}
				
				Por lo tanto, luego de reflejarse en los espejos, los campos quedan descriptos por 
				
				\begin{equation}
				\begin{cases}
				
				E'_x=R E_x\\
				E'_y=R E_y e^{-i(\Delta \phi - \delta_y)}
				\end{cases}
				\end{equation}
				
				calculando ahora el valor de $I'$ se obtiene que , desperdiciando las perdidas, $I'=I$.
				
				\subsection{Condiciones de borde}
				
				
			
				\begin{align}
				F_x(\eta,\tau)|_{\eta=0}&=R F_x(\eta,\tau)|_{\eta=1}\\
				F_y(\eta,\tau)|_{\eta=0}&=R F_y(\eta,\tau)|_{\eta=1} e^{-i\Delta \phi}
				\end{align}
				\label{eq: cc}
			
				
				donde 
				
				\begin{equation}
				\begin{cases}
					F_x(\eta,\tau)=E_x(\eta,\tau)e^{-\eta \ln(R)}\\
					\tilde{P_x}(\eta,\tau)=P_x e^{-\eta \ln(R)}\\			
					F_y(\eta,\tau)=E_y(\eta,\tau)e^{-\eta \ln(R)}e^{i\Delta \phi \eta}\\
					\tilde{P_y}(\eta,\tau)=P_y e^{-\eta \ln(R)}e^{i\Delta \phi \eta}
				\end{cases}
				\end{equation} 			
				
				Reemplazando en las ecuaciones \ref{eq: cc} se obtiene
				\begin{equation}
				E_x(0,\tau)=R E_x(1,\tau)e^{-ln(R)}
				\label{eq: cc2}
				\end{equation}
				
				condición de contorno de ring laser: $E(0,\tau)=E(1,\tau)$
				
		Finalmente se obtienen las ecuaciones que vamos a estudiar:
		
		\[
		\begin{cases}
		\partial_{\tau} E_x=-k E_x + \mu P_x \\
		\partial_{\tau} E_y=-k E_y + \mu P_y + i(\Delta \phi_0 + m.cos(w_{mod}\tau))E_y \\
		\partial_{\tau} P_{x,y}=-(1+i\delta)P_{x,y}+E_{x,y}D \\
		\partial_{\tau} D=-\gamma_{||}(D-D_0+\tfrac{1}{2}(E^*_{x,y}P_{x,y}+E_{x,y}P^*_{x,y})) \\
		\end{cases}
		\label{eq: final eqs 1}
		\]
		
%		
%		with $ E_{x,y}$ and $P_{x,y}$  $\in \mathbb{C}$
%		
		donde 
		\begin{itemize}
			\item $k$ es la razón entre el ancho de la cavidad (cavity linewidth) y el ancho espectral (atomic linewidth).
			\item $\mu$ es la ..
			\item $\delta $  es un parámetro de sintonizacion (detuning) de la cavidad.
			\item $\Delta \phi_0 $ es el  \textit{offset} de la modulación . La correlación física de $\Delta \phi_0 $ es la birrefringencia del medio, sumada al cambio de fase inducido por el modulador electro óptico .
			\item $m$ es la amplitud de la modulación . Físicamente ..
			\item $D_0$ 
			\item $\tau$ es el tiempo normalizado.
			\item $\gamma_{||}$ es taza de relajación para la inversión de población (relaxation rate) .
			\item $\gamma_{\bot}$ es ancho espectral (atomic lindewidth) .
		\end{itemize}
		
		
		Escribiendo los campos como parte real e imaginaria:
		
		\[
		\begin{cases}
		\partial_{\tau} Re(E_x)=-k Re(E_x) + \mu Re(P_x) \\
		\partial_{\tau} Im(E_x)=-k Im(E_x) + \mu Im(P_x) \\
		\partial_{\tau} Re(E_y)=-k Re(E_y) + \mu Re(P_y) -(\Delta \phi_0 + m.cos(w_{mod}.\tau)).Im(E_y) \\
		\partial_{\tau} Im(E_y)=-k Im(E_y) + \mu Im(P_y) + (\Delta \phi_0 + m.cos(w_{mod}.\tau)).Re(E_y) \\
		\partial_{\tau} Re(P_{x,y})=-(Re(P_{x,y})-Im(P_{x,y})\delta)+Re(E_{x,y}).D \\
		\partial_{\tau} Im(P_{x,y})=-(Im(P_{x,y})+Re(P_{x,y})\delta)+Im(E_{x,y}).D \\
		\partial_{\tau} D=-\gamma_{||}(D-D_0+(Re(E_{x,y})Re(P_{x,y})+Im(E_{x,y})Im(P_{x,y}))) \\
		\end{cases}
		\]
		
		\begin{lstlisting}
		def mb(y, t,k,mu,Dphi0,d,g,D0,m,wf):
		""" y[0],y[1] campo electrico en x. y[2],y[3] campo electrico en y, y[4],y[5]  polarizacion en x, y[6],y[7]  polarizacion en y, y[8] poblacion. """
		dfxr=-k*y[0]+mu*y[4]
		dfxi=-k*y[1]+mu*y[5]
		dfyr=-k*y[2]+mu*y[6]-y[3]*(Dphi0+m*np.cos(wf*t))
		dfyi=-k*y[3]+mu*y[7]+y[2]*(Dphi0+m*np.cos(wf*t))
		drxr=-(1*y[4]-d*y[5])+y[0]*y[8]
		drxi=-(1*y[5]+d*y[4])+y[1]*y[8]
		dryr=-(1*y[6]-d*y[7])+y[2]*y[8]
		dryi=-(1*y[7]+d*y[6])+y[3]*y[8]
		ddelta=-g*(y[8]-D0+(y[0]*y[4]+y[1]*y[5]+y[2]*y[6]+y[3]*y[7]))
		return [dfxr,dfxi,dfyr,dfyi,drxr,drxi,dryr,dryi,ddelta]
		\end{lstlisting}
		
		Normalizations made: 
		$\tau= \gamma_{\bot}.t$, $k=\tfrac{\bar{k}}{\gamma_{\bot}}$,  $\gamma_{\parallel}=\tfrac{\bar{\gamma_{\parallel}}}{\gamma_{\bot}}$, $\eta=\tfrac{z}{L}$, $\delta'_{ac}=\tfrac{w_a-w_0}{\gamma_{\bot}}$
		
		
		Aproximations: 
		
		1-$k,\gamma_{\parallel}<<\gamma_{\bot}$   -- Homogenously broadened laser linewidth $ \nabla^2 E-\frac{1}{c^2}\partial^2_{t}E=\alpha \partial^2_{t}E$
		
		2-Plane wave: $\nabla^2_{\bot}=0$
		
		3-Two level medium
		
		4-Slowly varying amplitud
		
		5-Unidirectional field
		
		6-Rotating wave approx $\partial_{t^2}<<\partial_t$
		
		7-Single longitudinal mode
		
		8-$g'->0$, $R_0->1$  -- Uniform field limit
		
		9-$m$,$w_{mod}<<1$, $w_{mod}<<\gamma_{\bot}$ % ..chequear..